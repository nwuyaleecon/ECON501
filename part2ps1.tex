%Jennifer Pan, August 2011

\documentclass[10pt,letter]{article}
	% basic article document class
	% use percent signs to make comments to yourself -- they will not show up.

\usepackage{enumitem}
\usepackage{amsmath}
\usepackage{amssymb}
	% packages that allow mathematical formatting

\usepackage{graphicx}
\usepackage{tikz}
	% package that allows you to include graphics

\usepackage{setspace}
	% package that allows you to change spacing

\onehalfspacing
	% text become 1.5 spaced

\usepackage{fullpage}
	% package that specifies normal margins


\begin{document}
	% line of code telling latex that your document is beginning


\title{ECON501 Problem Set 1}

\author{Nicholas Wu}

\date{Spring 2021}
	% Note: when you omit this command, the current dateis automatically included

\maketitle
	% tells latex to follow your header (e.g., title, author) commands.

\section*{Problem 2}
\begin{enumerate}[label=(\alph*)]
\item The seller problem is
\[ \max \sum_i p_i (t(\theta_i) - c(q(\theta_i))) \]
subject to IC:
\[ \forall i\neq j \ \ \theta_i v(q(\theta_i)) - t(\theta_i) \ge \theta_i v(q(\theta_j)) - t(\theta_j) \]
and IR:
\[ \forall i \ \ \theta_i v(q(\theta_i)) - t(\theta_i) \ge 0 \]
Now, manipulating IC, we get that for any $i > j$,
\[ \theta_i v(q(\theta_i)) - t(\theta_i) \ge \theta_i v(q(\theta_j)) - t(\theta_j) \]
\[ \theta_i (v(q(\theta_i)) - v(q(\theta_j))) \ge t(\theta_i)  - t(\theta_j) \]
Similarly from IC,
\[ \theta_j v(q(\theta_i)) - t(\theta_j) \ge \theta_j v(q(\theta_i)) - t(\theta_i) \]
\[ t(\theta_i) - t(\theta_j) \ge \theta_j (v(q(\theta_i)) -  v(q(\theta_i))) \]
Putting these two together, we get:
\[ \theta_i (v(q(\theta_i)) - v(q(\theta_j))) \ge t(\theta_i)  - t(\theta_j) \ge \theta_j (v(q(\theta_i)) -  v(q(\theta_i))) \]
\[ (\theta_i - \theta_j)(v(q(\theta_i)) - v(q(\theta_j))) \ge 0 \]
By our supposition, $i > j$, so $\theta_i > \theta_j$, and hence the first term in the product is positive. This implies the second term must also be positive, and hence
\[ v(q(\theta_i)) - v(q(\theta_j)) \ge 0 \]
\[ v(q(\theta_i)) \ge v(q(\theta_j)) \]
Since $v$ is monotonically increasing by assumption, this implies $q(\theta_i) \ge q(\theta_j)$. Hence $q$ is monotonic if IC holds.
\item
From IC, for $i > 1$, $\theta_i > \theta_1$, and hence we have
\[ \theta_i v(q(\theta_i)) - t(\theta_i) \ge \theta_i v(q(\theta_1)) - t(\theta_1) > \theta_1 v(q(\theta_1)) - t(\theta_1)  \]
But the expression on the right is $> 0$ by IR, hence IR for all $i > 1$ are redundant.

Additionally, we note that from our previous formulation of IC in part $a$, we have for $i \neq j$,
\[ \theta_i v(q(\theta_i)) - t(\theta_i) \ge \theta_i v(q(\theta_j)) - t(\theta_j) \]
\[ \theta_i (v(q(\theta_i)) - v(q(\theta_j))) \ge t(\theta_i)  - t(\theta_j) \]
We claim that IC for consecutive $i,j$ is sufficient. We show this for the case where $i > j$ (the case where $i < j$ is similar, using the opposite directional IC constraint). Then we have from the consecutive IC ($i-j = \pm 1$):
\[ \theta_i (v(q_i) - v(q_{i-1})) \ge t_i - t_{i-1} \]
\[ \theta_{i-1} (v(q_{i-1}) - v(q_{i-2})) \ge t_{i-1} - t_{i-2} \]
\[ \vdots \]
\[ \theta_{j+1} (v(q_{j+1}) - v(q_{j})) \ge t_{j+1} - t_{j} \]
Summing, we get
\[ \sum_{k = j+1}^{i} \theta_k (v(q_k) - v(q_{k-1})) \ge t_i - t_j  \]
Since the RHS telescopes. But since $i \ge k$, we have $\theta_i \ge \theta_k$, and hence
\[ \sum_{k = j+1}^{i} \theta_i (v(q_k) - v(q_{k-1})) \ge \sum_{k = j+1}^{i} \theta_k (v(q_k) - v(q_{k-1})) \ge t_i - t_j  \]
But the LHS telescopes, and we get
\[ \theta_i (v(q_i) - v(q_j)) \ge t_i - t_j \]
and hence we have shown IC for $i > j$ from consecutive IC.
So the only non-redundant constraints are consecutive IC and IR for $1$.
\item IR1 must bind (otherwise we can increase all transfers by $\epsilon$). Now, consider the constraint IC for $k, k-1$:
\[ \theta_k v(q_k) - t_k \ge \theta_{k} v(q_{k-1}) - t_{k-1} \]
Suppose this did not bind. Then consider increasing the transfers $t_k, t_{k+1}, t_{k+2} ... t_n$ by $\epsilon$. Clearly, this doesn't break any of the ICs above $k$ or below $k-1$. Clearly, IC for $k, k-1$ still holds as long as $\epsilon$ is small. Additionally, IC for $k-1, k$ is
\[ \theta_{k-1} v(q_{k-1}) - t_{k-1} \ge \theta_{k-1} v(q_k) - t_k  \]
so increasing $t_k$ without changing $t_{k-1}$ maintains this constraint. Hence, we have IC $k, k-1$ must bind. Lastly, since $q$ is monotonic in $\theta$, we have $v(q_{k-1}) - v(q_k) \le 0$, and hence since IC $k, k-1$ binds,
\[ \theta_{k-1} (v(q_{k-1}) - v(q_k)) \ge \theta_k(v(q_{k-1}) - v(q_k)) = t_{k-1} - t_k \]
\[ \theta_{k-1} v(q_{k-1}) - t_{k-1} \ge \theta_{k-1} v(q_k) - t_k\]
so IC $k-1, k$ also holds (but does not necessarily bind).
\end{enumerate}
\section*{Problem 3}
From class, we know that
\[ q(\theta) = \arg\max v(q) \psi(\theta) - c(q)  = \arg\max v(q) \psi(\theta) - q \]
We know that in order for this FOC to be valid, we need $\psi(\theta) > 0$, and hence under the regularity assumption, there exists a unique $\theta^*$ such that
\[ \psi(\theta^*) = 0 \iff \theta^* - \frac{1-F(\theta^*)}{f(\theta^*)} = 0\]
Since regularity implies $\psi$ is increasing in $\theta$, we have for $\theta \le \theta^*$, $q(\theta) = 0$, and for $\theta > \theta^*$, we have the FOC
\[ v'(q) \psi(\theta) = 1 \]
\[ v'(q(\theta)) = 1/\psi(\theta) \]
Also, we know that
\[ t(\theta)= \theta v(q(\theta)) - \int_{0}^\theta v(q(x)) \ dx \]
\[ t'(\theta) = v(q(\theta)) + \theta v'(q(\theta))q'(\theta) - v(q(\theta)) = v'(q(\theta))q'(\theta) = \frac{\theta }{\psi(\theta)}q'(\theta) \]
Hence
\[ t(\theta) = t(0) + \int_0^{\theta} \frac{x}{\psi(x)}q'(x) \ dx \]
Since $t(0) = 0 v(q(0)) = 0$,
\[ t(\theta) = \int_0^{\theta} \frac{x}{\psi(x)}q'(x) \ dx \]
Integrating the RHS by parts,
\[ t(\theta) = \frac{x}{\psi(x)}q(x) \Bigr|_0^\theta - \int_0^\theta \frac{\psi(x) - x \psi'(x)}{\psi(x)^2 }q(x) \ dx \]
\[ t(\theta) = \frac{\theta}{\psi(\theta)}q(\theta) - \int_0^\theta \frac{\psi(x) - x \psi'(x)}{\psi(x)^2 }q(x) \ dx \]
Since $q(0) = 0$. Dividing by $q(\theta)$, we get
\[ \frac{t(\theta)}{q(\theta)} = \frac{\theta}{\psi(\theta)} - \frac{1}{q(\theta)}\int_0^\theta \frac{\psi(x) - x \psi'(x)}{\psi(x)^2 }q(x) \ dx \]
Taking the derivative wrt $\theta$, we get
\[ \frac{\partial}{\partial \theta}\frac{t(\theta)}{q(\theta)} = \frac{\psi(\theta) - \theta \psi'(\theta)}{(\psi(\theta))^2} + \frac{q'(\theta)}{q(\theta)^2}\int_0^\theta \frac{\psi(x) - x\psi'(x)}{\psi(x)^2 }q(x) \ dx - \frac{\psi(\theta) - \theta \psi'(\theta)}{\psi(\theta)^2 } \]
\[ = \frac{q'(\theta)}{q(\theta)^2}\int_0^\theta \frac{\psi(x) - x\psi'(x)}{\psi(x)^2 }q(x) \ dx \]
Now, we know $q' > 0$, $q^2 > 0$, $\psi^2 > 0$. Then $\psi(\theta) \le \theta \psi'(\theta) $ is a sufficient condition for this expression to be negative (equivalently, $\theta / \psi(\theta)$ is decreasing).
\section*{Problem 4}
\begin{enumerate}[label=(\alph*)]
\item
\item
\end{enumerate}
\section*{Problem 5}
\begin{enumerate}[label=(\alph*)]
\item Suppose agent $i$ realizes type $\theta_i$. By truthful reporting, the expected payout is given by the probability of winning the good times the expected payout given the good was won:
\[ \theta_i \left(\theta_i - 2 * (\theta_i/2)\right) = \theta_i (0) = 0 \]
However, by reporting some $\theta_i - \epsilon$, the expected payout is then
\[ (\theta_i - \epsilon) \left(\theta_i  - 2 * ((\theta_i - \epsilon)/2)\right) = (\theta_i - \epsilon) (\epsilon) > 0 \]
Hence truthful reporting cannot be an equilibrium, since both players gain strictly higher expected payoff by underreporting.
\item
\end{enumerate}
\section*{Problem 6}
\begin{enumerate}[label=(\alph*)]
\item
\item
\end{enumerate}
\section*{Problem 7}
\section*{Problem 8}
\end{document}
	% line of code telling latex that your document is ending. If you leave this out, you'll get an error
