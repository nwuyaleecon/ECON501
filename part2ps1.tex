%Jennifer Pan, August 2011

\documentclass[10pt,letter]{article}
	% basic article document class
	% use percent signs to make comments to yourself -- they will not show up.

\usepackage{enumitem}
\usepackage{amsmath}
\usepackage{amssymb}
	% packages that allow mathematical formatting

\usepackage{graphicx}
\usepackage{tikz}
	% package that allows you to include graphics

\usepackage{setspace}
	% package that allows you to change spacing

\onehalfspacing
	% text become 1.5 spaced

\usepackage{fullpage}
	% package that specifies normal margins


\begin{document}
	% line of code telling latex that your document is beginning


\title{ECON501 Problem Set 1}

\author{Nicholas Wu}

\date{Spring 2021}
	% Note: when you omit this command, the current dateis automatically included

\maketitle
	% tells latex to follow your header (e.g., title, author) commands.

\section*{Problem 2}
\begin{enumerate}[label=(\alph*)]
\item The seller problem is
\[ \max \sum_i p_i (t(\theta_i) - c(q(\theta_i))) \]
subject to IC:
\[ \forall i\neq j \ \ \theta_i v(q(\theta_i)) - t(\theta_i) \ge \theta_i v(q(\theta_j)) - t(\theta_j) \]
and IR:
\[ \forall i \ \ \theta_i v(q(\theta_i)) - t(\theta_i) \ge 0 \]
Now, manipulating IC, we get that for any $i > j$,
\[ \theta_i v(q(\theta_i)) - t(\theta_i) \ge \theta_i v(q(\theta_j)) - t(\theta_j) \]
\[ \theta_i (v(q(\theta_i)) - v(q(\theta_j))) \ge t(\theta_i)  - t(\theta_j) \]
Similarly from IC,
\[ \theta_j v(q(\theta_i)) - t(\theta_j) \ge \theta_j v(q(\theta_i)) - t(\theta_i) \]
\[ t(\theta_i) - t(\theta_j) \ge \theta_j (v(q(\theta_i)) -  v(q(\theta_i))) \]
Putting these two together, we get:
\[ \theta_i (v(q(\theta_i)) - v(q(\theta_j))) \ge t(\theta_i)  - t(\theta_j) \ge \theta_j (v(q(\theta_i)) -  v(q(\theta_i))) \]
\[ (\theta_i - \theta_j)(v(q(\theta_i)) - v(q(\theta_j))) \ge 0 \]
By our supposition, $i > j$, so $\theta_i > \theta_j$, and hence the first term in the product is positive. This implies the second term must also be positive, and hence
\[ v(q(\theta_i)) - v(q(\theta_j)) \ge 0 \]
\[ v(q(\theta_i)) \ge v(q(\theta_j)) \]
Since $v$ is monotonically increasing by assumption, this implies $q(\theta_i) \ge q(\theta_j)$. Hence $q$ is monotonic if IC holds.
\item
From IC, for $i > 1$, $\theta_i > \theta_1$, and hence we have
\[ \theta_i v(q(\theta_i)) - t(\theta_i) \ge \theta_i v(q(\theta_1)) - t(\theta_1) > \theta_1 v(q(\theta_1)) - t(\theta_1)  \]
But the expression on the right is $> 0$ by IR, hence IR for all $i > 1$ are redundant.

Additionally, we note that from our previous formulation of IC in part $a$, we have for $i \neq j$,
\[ \theta_i v(q(\theta_i)) - t(\theta_i) \ge \theta_i v(q(\theta_j)) - t(\theta_j) \]
\[ \theta_i (v(q(\theta_i)) - v(q(\theta_j))) \ge t(\theta_i)  - t(\theta_j) \]
We claim that IC for consecutive $i,j$ is sufficient. We show this for the case where $i > j$ (the case where $i < j$ is similar, using the opposite directional IC constraint). Then we have from the consecutive IC ($i-j = \pm 1$):
\[ \theta_i (v(q_i) - v(q_{i-1})) \ge t_i - t_{i-1} \]
\[ \theta_{i-1} (v(q_{i-1}) - v(q_{i-2})) \ge t_{i-1} - t_{i-2} \]
\[ \vdots \]
\[ \theta_{j+1} (v(q_{j+1}) - v(q_{j})) \ge t_{j+1} - t_{j} \]
Summing, we get
\[ \sum_{k = j+1}^{i} \theta_k (v(q_k) - v(q_{k-1})) \ge t_i - t_j  \]
Since the RHS telescopes. But since $i \ge k$, we have $\theta_i \ge \theta_k$, and hence
\[ \sum_{k = j+1}^{i} \theta_i (v(q_k) - v(q_{k-1})) \ge \sum_{k = j+1}^{i} \theta_k (v(q_k) - v(q_{k-1})) \ge t_i - t_j  \]
But the LHS telescopes, and we get
\[ \theta_i (v(q_i) - v(q_j)) \ge t_i - t_j \]
and hence we have shown IC for $i > j$ from consecutive IC.
So the only non-redundant constraints are consecutive IC and IR for $1$.
\item IR1 must bind (otherwise we can increase all transfers by $\epsilon$). Now, consider the constraint IC for $k, k-1$:
\[ \theta_k v(q_k) - t_k \ge \theta_{k} v(q_{k-1}) - t_{k-1} \]
Suppose this did not bind. Then consider increasing the transfers $t_k, t_{k+1}, t_{k+2} ... t_n$ by $\epsilon$. Clearly, this doesn't break any of the ICs above $k$ or below $k-1$. Clearly, IC for $k, k-1$ still holds as long as $\epsilon$ is small. Additionally, IC for $k-1, k$ is
\[ \theta_{k-1} v(q_{k-1}) - t_{k-1} \ge \theta_{k-1} v(q_k) - t_k  \]
so increasing $t_k$ without changing $t_{k-1}$ maintains this constraint. Hence, we have IC $k, k-1$ must bind. Lastly, since $q$ is monotonic in $\theta$, we have $v(q_{k-1}) - v(q_k) \le 0$, and hence since IC $k, k-1$ binds,
\[ \theta_{k-1} (v(q_{k-1}) - v(q_k)) \ge \theta_k(v(q_{k-1}) - v(q_k)) = t_{k-1} - t_k \]
\[ \theta_{k-1} v(q_{k-1}) - t_{k-1} \ge \theta_{k-1} v(q_k) - t_k\]
so IC $k-1, k$ also holds (but does not necessarily bind).
\end{enumerate}
\section*{Problem 3}
From class, we know that
\[ q(\theta) = \arg\max v(q) \psi(\theta) - c(q)  = \arg\max v(q) \psi(\theta) - q \]
We know that in order for this FOC to be valid, we need $\psi(\theta) > 0$, and hence under the regularity assumption, there exists a unique $\theta^*$ such that
\[ \psi(\theta^*) = 0 \iff \theta^* - \frac{1-F(\theta^*)}{f(\theta^*)} = 0\]
Since regularity implies $\psi$ is increasing in $\theta$, we have if $\theta \le \theta^*$, $q(\theta) = 0$, and othehrwise for $\theta > \theta^*$, we have the FOC
\[ v'(q) \psi(\theta) = 1 \]
\[ v'(q(\theta)) = 1/\psi(\theta) \]
Also, we know that
\[ t(\theta)= \theta v(q(\theta)) - \int_{0}^\theta v(q(x)) \ dx \]
\[ t'(\theta) = v(q(\theta)) + \theta v'(q(\theta))q'(\theta) - v(q(\theta)) = v'(q(\theta))q'(\theta) = \frac{\theta }{\psi(\theta)}q'(\theta) \]
Hence
\[ t(\theta) = t(0) + \int_0^{\theta} \frac{x}{\psi(x)}q'(x) \ dx \]
Since $t(0) = 0 v(q(0)) = 0$,
\[ t(\theta) = \int_0^{\theta} \frac{x}{\psi(x)}q'(x) \ dx \]
Integrating the RHS by parts,
\[ t(\theta) = \frac{x}{\psi(x)}q(x) \Bigr|_0^\theta - \int_0^\theta \frac{\psi(x) - x \psi'(x)}{\psi(x)^2 }q(x) \ dx \]
\[ t(\theta) = \frac{\theta}{\psi(\theta)}q(\theta) - \int_0^\theta \frac{\psi(x) - x \psi'(x)}{\psi(x)^2 }q(x) \ dx \]
Since $q(0) = 0$. Dividing by $q(\theta)$, we get
\[ \frac{t(\theta)}{q(\theta)} = \frac{\theta}{\psi(\theta)} - \frac{1}{q(\theta)}\int_0^\theta \frac{\psi(x) - x \psi'(x)}{\psi(x)^2 }q(x) \ dx \]
Taking the derivative wrt $\theta$, we get
\[ \frac{\partial}{\partial \theta}\frac{t(\theta)}{q(\theta)} = \frac{\psi(\theta) - \theta \psi'(\theta)}{(\psi(\theta))^2} + \frac{q'(\theta)}{q(\theta)^2}\int_0^\theta \frac{\psi(x) - x\psi'(x)}{\psi(x)^2 }q(x) \ dx - \frac{\psi(\theta) - \theta \psi'(\theta)}{\psi(\theta)^2 } \]
\[ = \frac{q'(\theta)}{q(\theta)^2}\int_0^\theta \frac{\psi(x) - x\psi'(x)}{\psi(x)^2 }q(x) \ dx \]
Now, we know $q' > 0$, $q^2 > 0$, $\psi^2 > 0$. Then $\psi(\theta) \le \theta \psi'(\theta) $ is a sufficient condition for this expression to be negative, since this makes the integrand negative at all values (equivalently, we can require $\theta / \psi(\theta)$ is decreasing).
\section*{Problem 4}
\begin{enumerate}[label=(\alph*)]
\item The regulator maximizes:
\[ \int_0^q p(x) \ dx - p(q) q + \alpha \Pi(q) - s\]
subject to
\[ p(q)q - C(q,\theta) + s \ge 0 \]
Note that given a $q$, we want to pick $s$ as small as possible to make the condition bind. Hence
\[ s = C(q,\theta)- p(q) q  \]
so $\Pi(q) = 0$. So the unconstrained maximization is given by
\[ \max_q \int_0^q p(x) \ dx - p(q) q -C(q,\theta) + p(q) q = \max_q \int_0^q p(x) \ dx - C(q, \theta) \]
The interior FOC of the relaxed problem is
\[ p(q) - \theta = 0 \]
\[ 1 - 2q = \theta \]
\[ q = \frac{1-\theta}{2} \]
\item The monopoly participation constraint is:
\[ p(q(\theta)) q(\theta) - K - \theta q(\theta) + s(\theta) \ge 0 \]
The IC constraints are then
\[ p(q(\theta)) q(\theta) - K - \theta q(\theta) + s(\theta) \ge p(q(\theta')) q(\theta') - K - \theta q(\theta') + s(\theta') \]
Note for $\theta < \overline{\theta}$ we get:
\[ p(q(\overline{\theta}))q(\overline{\theta}) - K - \overline{\theta} q(\overline{\theta}) \le p(q(\overline{\theta}))q(\overline{\theta}) - K - \theta q(\overline{\theta})  \le p(q(\theta))q({\theta}) - K - \theta q({\theta})   \]
Hence IR is redundant except for $\overline{\theta}$. Note that this IR must bind, else we can uniformly lower the subsidies by the same amount. So $U(\overline{\theta}) = 0$. IC can be rewritten
\[ U(\theta) = \max_{\theta'} p(q(\theta')) q(\theta') - K - \theta q(\theta') + s(\theta')  \]
By the envelope theorem, $U'(\theta) = -q(\theta)$. So \[ U(\theta) = \int_{\theta}^{\overline{\theta}} q(x) \ dx \]
Then we can determine subsidies:
\[ U(\theta) = \int_{\theta}^{\overline{\theta}} q(x) \ dx = p(q(\theta)) q(\theta) - K - \theta q(\theta) + s(\theta)  \]
\[ s(\theta) = \int_{\theta}^{\overline{\theta}} q(x) \ dx - p(q(\theta)) q(\theta) + K + \theta q(\theta) =  \int_{\theta}^{\overline{\theta}} q(x) \ dx - p(q(\theta)) q(\theta) + C(q, \theta) \]
Then the maximization problem becomes:
\[ \max \int_{\underline{\theta}}^{\overline{\theta}} \left(V(q(\theta)) + \alpha \Pi(q(\theta)) - s(\theta)\right)f(\theta) \ d\theta \]
\[ = \max \int_{\underline{\theta}}^{\overline{\theta}} \left(\int_0^{q(\theta)}p(x) \ dx - p(q(\theta))q(\theta) + \alpha \left(\int_{\theta}^{\overline{\theta}} q(x) \ dx \right) - \left(\int_{\theta}^{\overline{\theta}} q(x) \ dx- p(q(\theta)) q(\theta) + K + \theta q(\theta) \right)\right)f(\theta) \ d\theta \]
\[ = \max_q \int_{\underline{\theta}}^{\overline{\theta}} \left(\int_0^{q(\theta)}p(x) \ dx -(1- \alpha) \left(\int_{\theta}^{\overline{\theta}} q(x) \ dx \right) - K - \theta q(\theta)\right)f(\theta) \ d\theta \]
\[ = \max_q \int_{\underline{\theta}}^{\overline{\theta}} \left(\int_0^{q(\theta)}p(x) \ dx - K - \theta q(\theta)\right)f(\theta) \ d\theta -(1- \alpha) \int_{\underline{\theta}}^{\overline{\theta}}\left(\int_{\theta}^{\overline{\theta}} q(x) \ dx \right) f(\theta) d \theta \]
\[ = \max_q \int_{\underline{\theta}}^{\overline{\theta}} \left(\int_0^{q(\theta)}p(x) \ dx - K - \theta q(\theta)\right)f(\theta) \ d\theta -(1- \alpha) \int_{\underline{\theta}}^{\overline{\theta}}\int_{\theta}^{\overline{\theta}} q(x) f(\theta) \ dx  \ d \theta \]
\[ = \max_q \int_{\underline{\theta}}^{\overline{\theta}} \left(\int_0^{q(\theta)}p(x) \ dx - K - \theta q(\theta)\right)f(\theta) \ d\theta - (1- \alpha) \int_{\underline{\theta}}^{\overline{\theta}} \int_{\underline{\theta}}^{x} q(x) f(\theta) \ d \theta \ dx    \]
\[ = \max_q \int_{\underline{\theta}}^{\overline{\theta}} \left(\int_0^{q(\theta)}p(x) \ dx - K - \theta q(\theta)\right)f(\theta) \ d\theta - (1- \alpha) \int_{\underline{\theta}}^{\overline{\theta}} q(x) F(x) \ dx    \]
\[ = \max_q \int_{\underline{\theta}}^{\overline{\theta}} \left(\int_0^{q(\theta)}p(x) \ dx - K - \theta q(\theta)\right)f(\theta) \ d\theta - (1- \alpha) \int_{\underline{\theta}}^{\overline{\theta}} q(x) \frac{F(x)}{f(x)} f(x0) \ dx    \]
\[ = \max_q \int_{\underline{\theta}}^{\overline{\theta}} \left(\int_0^{q(\theta)}p(x) \ dx - K - \theta q(\theta) - (1-\alpha)\frac{F(\theta)}{f(\theta)}q(\theta) \right)f(\theta) \ d\theta    \]
\[ = \max_q \int_{\underline{\theta}}^{\overline{\theta}} \left(\int_0^{q(\theta)}p(x) \ dx - K - \psi(\theta) q(\theta)  \right)f(\theta) \ d\theta    \]
where
\[ \psi(\theta) = \theta + (1-\alpha) \frac{F(\theta)}{f(\theta)} \]
Since $\theta + F(\theta)/f(\theta)$ is increasing, we have
\[ 1 + \frac{\partial}{\partial \theta} \frac{F(\theta)}{f(\theta)} > 0 \]
\[ \frac{\partial}{\partial \theta} \frac{F(\theta)}{f(\theta)} > -1 \]
\[ (1-\alpha)\frac{\partial}{\partial \theta} \frac{F(\theta)}{f(\theta)} > -(1-\alpha) \]
\[ 1 + (1-\alpha)\frac{\partial}{\partial \theta} \frac{F(\theta)}{f(\theta)} > \alpha > 0 \]
Hence $\theta + (1-\alpha)F(\theta)/f(\theta)$ is also increasing. Now, we just have to maximize:
\[\int_0^{q(\theta)}p(x) \ dx - K - \psi(\theta) q(\theta)  \]
at each $\theta$. The FOC gives
\[ p(q(\theta)) - \psi(\theta) = 0 \]
\[ 1 - 2q(\theta) - \psi(\theta) = 0 \]
\[ q(\theta) = \frac{1-\psi(\theta)}{2}\]
This interior solution is only valid as long as the integrand is positive, or
\[ \int_0^{q(\theta)}p(x) \ dx - K - \psi(\theta) q(\theta) \ge 0 \]
\[ K \le \int_0^{q(\theta)}p(x) \ dx - \psi(\theta) q(\theta) = q(\theta) - q(\theta)^2 - \psi(\theta)q(\theta) = \frac{(1-\psi(\theta))^2}{4} \]
But by assumption, $K \le \frac{(1-\underline{\theta})^2}{4} = \frac{(1-\psi(\underline{\theta}))^2}{4}$. Since $\psi$ is increasing, there exists some $\theta^*$ such that $K = (1-\psi(\theta^*))^2/4$. For $\theta \ge \theta^*$, the regulator sets $q = 0$, and for $\theta < \theta^*$, the regulator sets $q = (1-\psi(\theta))/2$.
This is almost the case of first best, but has $\psi(\theta)$ instead of $\theta$; that is, the regulator is forced to decrease quantity in order to properly incentivize truthful reporting of $\theta$ by the monopoly.
\end{enumerate}
\section*{Problem 5}
\begin{enumerate}[label=(\alph*)]
\item Suppose agent $i$ realizes type $\theta_i$. By truthful reporting, the expected payout is given by the probability of winning the good times the expected payout given the good was won:
\[ \theta_i \left(\theta_i - 2 * (\theta_i/2)\right) = \theta_i (0) = 0 \]
However, by reporting some $\theta_i - \epsilon$, the expected payout is then
\[ (\theta_i - \epsilon) \left(\theta_i  - 2 * ((\theta_i - \epsilon)/2)\right) = (\theta_i - \epsilon) (\epsilon) > 0 \]
Hence truthful reporting cannot be an equilibrium, since both players gain strictly higher expected payoff by underreporting.
\item Fix the player 2 report as $b_2$. Player 1 only wants the good iff $\theta_1 \ge 2b_2$, or $\theta_1/2 \ge b_2$. Hence, it is optimal to bid $\theta_1/2$. Symmetrically, it is optimal for player 2 to bid $\theta_2/2$. Hence, each player bids half of his/her own true value. Note it is ex-post efficient, since the player with the highest type gets the good. The truthful-reporting direct mechanism that implements this same ex-post allocation is then just the standard second-price auction.
\end{enumerate}
\section*{Problem 6}
\begin{enumerate}[label=(\alph*)]
\item In the direct mechanism, each player reports his/her type. We need to specify the allocation rule that induces maximal revenue. We sell to the agent with the highest virtual type. The virtual type of player 1 is given by:
\[ \psi_1(\theta_1) = \theta_1 - \frac{1 - (\theta_1 - 1)}{1} = 2\theta_1 - 2 \]
The virtual type of 2 is
\[ \psi_2(\theta_2) = \theta_2 - \frac{1 - (1/2)(\theta_2 - 1)}{1/2} = 2\theta_2 - 3 \]
Note that $\psi_1(\theta_1) \ge 0$ always, and $\psi_2(\theta) > 0$ for $\theta_2 > 1.5$. Hence we want to give to player 2 if $\psi_2 > \psi_1$, else give to player 1. This condition is also rewriteable as:
\[ 2\theta_2 - 3 > 2\theta_1 - 2 \]
\[ \theta_2 > \theta_1 + \frac{1}{2}\]
\item We present a BNE such that the auction implements the same allocation rule. Consider the strategy for player 1, with $b_2$ fixed. Player 1 wants the good iff $\theta_1 > b_2 - \frac{1}{2}$, or $\theta_1+ (1/2) \ge b_2$. Hence, it is optimal to bid $\theta_1 + 1/2$. Now, for player 2, fixing $b_1$, player 2 wants the good iff $\theta_2 \ge b_1$, and hence it is optimal to bid $\theta_2$. So player 1 bids $\theta_1 + 1/2$, and player 2 bids $\theta_2$, and player 2 gets the good iff $\theta_2 > \theta_1 + (1/2)$, exactly the revenue maximizing result we derived in the previous part.
\end{enumerate}
\section*{Problem 7}
Let $G(x)$ denote the CDF of the the highest bid of $N-1$ players. That is,
\[ G(x) = (F(x))^{N-1} \]
We denote the pdf assicated with $G$ as $g$. Then we know by the revenue equivalence theorem, the interim expected payoff is
\[ \int_{\underline{v}}^{v_i} G(x) \ dx = G(v_i) v_i + (1-G(v_i))(- b(v_i))  \]
Solving for $b(v_i)$, we get
\[ (1-G(v_i)) b(v_i) = G(v_i) v_i - \int_{\underline{v}}^{v_i} G(x) \ dx    \]
\[  b(v_i) = \frac{G(v_i)}{1-G(v_i)} v_i -\frac{1}{1-G(v_i)} \int_{\underline{v}}^{v_i} G(x) \ dx    \]
Integrating by parts, we get
\[  b(v_i) = \frac{G(v_i)}{1-G(v_i)} v_i -\frac{1}{1-G(v_i)} \int_{\underline{v}}^{v_i} G(x) \ dx    \]
\[ = \frac{G(v_i)}{1-G(v_i)} v_i - \frac{1}{1-G(v_i)}  \left( v_i G(v_i) - \int_{\underline{v}}^{v_i} x g(x) \ dx  \right)  \]
\[ =  \int_{\underline{v}}^{v_i} x g(x) \ dx   \]
Hence players bidding according to this function gives a symmetric equilibrium.
\section*{Problem 8}
Once again, let $G(x)$ denote the CDF of the highest bid of $N-1$ players as in the previous problem, and let the pdf associated with $G$ be $g$. By revenue equivalence, the interim expected payoff of player with value $v_i$ is
\[ \int_{0}^{v_i} G(x) = G(v_i) \left( v_i - E(\alpha s(v_i) + (1-\alpha) s(\max v_{-i}) | v_i = \max v)\right)  \]
\[ \int_{0}^{v_i} G(x) = G(v_i) \left( v_i - \alpha s(v_i)  - (1-\alpha) E( s(\max v_{-i}) | v_i = \max v)\right)  \]
\[ \int_{0}^{v_i} G(x) = (v_i - \alpha s(v_i))G(v_i) - (1-\alpha) \int_0^{v_i}  s(x) g(x) \ dx \]
Differentiating both sides wrt $v_i$, we get
\[ G(v_i) = (v_i - \alpha s(v_i))g(v_i) + (1 - \alpha s'(v_i))G(v_i)  - (1-\alpha)s(v_i) g(v_i)  \]
\[ G(v_i) = v_i g(v_i) - \alpha s(v_i) g(v_i) + G(v_i) - \alpha G(v_i) s'(v_i) - (1-\alpha) s(v_i) g(v_i)  \]
\[ 0 = v_i g(v_i)  - \alpha G(v_i) s'(v_i) -  s(v_i) g(v_i)  \]
\[ \alpha G(v_i) s'(v_i) + s(v_i) g(v_i)  = v_i g(v_i)    \]
\[  s'(v_i) + s(v_i) \frac{g(v_i)}{\alpha G(v_i)}  = v_i \frac{g(v_i)}{\alpha G(v_i)}    \]
Let \[ \varphi(x) = e^{\int_{0}^x \frac{g(t)}{\alpha G(t)} \ dt}  \]
Then \[ \varphi'(x) = \frac{g(t)}{\alpha G(t)} \varphi(x) \]
Then
\[  s'(v_i)\varphi(v_i) + s(v_i) \frac{g(v_i)}{\alpha G(v_i)} \varphi(v_i) = v_i \frac{g(v_i)}{\alpha G(v_i)} \varphi(v_i)   \]
\[  s'(v_i)\varphi(v_i) + s(v_i) \varphi' (v_i) = v_i \varphi'(v_i)   \]
\[ \frac{\partial}{\partial v_i} s(v_i)\varphi(v_i) = v_i \varphi'(v_i) \]
Integrating, we get
\[ s(v_i)\varphi(v_i) = 0(\varphi'(0)) + \int_{0}^{v_i} x \varphi'(x) \ dx  \]
\[ s(v_i)\varphi(v_i) = \int_{0}^{v_i} x \varphi'(x) \ dx  \]
\[ s(v_i)\varphi(v_i) = x \varphi(x) \Bigr|_0^{v_i} - \int_{0}^{v_i} \varphi(x) \ dx  \]
\[ s(v_i)\varphi(v_i) = v_i \varphi(v_i)  - \int_{0}^{v_i} \varphi(x) \ dx  \]
\[ s(v_i) = v_i  - \frac{1}{\varphi(v_i)}\int_{0}^{v_i} \varphi(x) \ dx  \]
where we integrated by parts. Now, we notice that we can rewrite $\varphi:$
\[ \varphi(x) = e^{\int_{0}^x \frac{g(t)}{\alpha G(t)} \ dt} = e^{\frac{1}{\alpha} (\log G(x) - \log G(0))} = e^{\log(G(x))/\alpha} = G(x)^{1/\alpha} = F(x)^{(N-1)/\alpha}  \]
So we can plug in
\[ s(v_i) = v_i  - \frac{1}{F(v_i)^{(N-1)/\alpha}}\int_{0}^{v_i} F(x)^{(N-1)/\alpha} \ dx  \]




\end{document}
	% line of code telling latex that your document is ending. If you leave this out, you'll get an error
