%Jennifer Pan, August 2011

\documentclass[10pt,letter]{article}
	% basic article document class
	% use percent signs to make comments to yourself -- they will not show up.

\usepackage{amsmath}
\usepackage{amssymb}
\usepackage{enumitem}
	% packages that allow mathematical formatting

\usepackage{graphicx}
\usepackage{tikz}
	% package that allows you to include graphics

\usepackage{setspace}
	% package that allows you to change spacing

\onehalfspacing
	% text become 1.5 spaced

\usepackage{fullpage}
	% package that specifies normal margins


\begin{document}
	% line of code telling latex that your document is beginning


\title{ECON501 Problem Set 6}

\author{Nicholas Wu}

\date{Spring 2021}
	% Note: when you omit this command, the current dateis automatically included

\maketitle
	% tells latex to follow your header (e.g., title, author) commands.

\section*{Problem 2}
\paragraph*{8.E.1} We write out the normal form of this game. The pure strategies are $AA$, $AN$, $NA$, $NN$, where $A$ means attack and $N$ means not attack, and the first letter denotes the action if the general sees strong type.
\begin{center}
\begin{tabular}{|c|c|c|c|c|}
\hline
& AA & AN & NA & NN \\ \hline
AA & $\frac{M}{4} - \frac{s+w}{2}, \frac{M}{4} - \frac{s+w}{2}$ & $\frac{M}{2} - \frac{s+w}{4}, \frac{M}{4} - \frac{s}{2} $ & $ \frac{3M}{4} - \frac{s+w}{4}, -\frac{w}{2}$ & $M, 0$ \\ \hline
AN & $ \frac{M}{4} - \frac{s}{2} , \frac{M}{2} - \frac{s+w}{4}$ & $\frac{M}{4} - \frac{s}{4}, \frac{M}{4} - \frac{s}{4}$ & $ \frac{M}{2} - \frac{s}{4}, \frac{M}{4} - \frac{w}{4}$ & $\frac{M}{2}, 0$ \\ \hline
NA & $ -\frac{w}{2}, \frac{3M}{4} - \frac{s+w}{4}$ & $\frac{M}{4} - \frac{w}{4}, \frac{M}{2} - \frac{s}{4}$ & $ \frac{M}{4} - \frac{w}{4},\frac{M}{4} - \frac{w}{4} $ & $\frac{M}{2}, 0$ \\ \hline
NN & $0, M$ & $0, \frac{M}{2}$ & $0, \frac{M}{2}$ & $0,0$ \\ \hline
\end{tabular}
\end{center}
We note $NA$ is strictly dominated by $AN$ since $w > s$, so we can restrict our attention to
\begin{center}
\begin{tabular}{|c|c|c|c|}
\hline
& AA & AN & NN \\ \hline
AA & $\frac{M}{4} - \frac{s+w}{2}, \frac{M}{4} - \frac{s+w}{2}$ & $\frac{M}{2} - \frac{s+w}{4}, \frac{M}{4} - \frac{s}{2} $ & $M, 0$ \\ \hline
AN & $ \frac{M}{4} - \frac{s}{2} , \frac{M}{2} - \frac{s+w}{4}$ & $\frac{M}{4} - \frac{s}{4}, \frac{M}{4} - \frac{s}{4}$ & $\frac{M}{2}, 0$ \\ \hline
NN & $0, M$ & $0, \frac{M}{2}$ & $0,0$ \\ \hline
\end{tabular}
\end{center}
If $M < s$, then $NN, NN$ is the only Nash equilibrium.

Note that the best reply to $NN$ is always $AA$. If $M/2 < s$ then the best reply to $AA$ is $NN$. So if $M/2 < s$, then $(AA, NN)$ and $(NN, AA)$ are Nash equilibria.

If $w > M > s$, then the best reply to $AN$ is $AN$. Hence, if this holds (note this is not mutually exclusive with $M/2 < s$), then $(AN, AN)$ is a Nash equilibrium.

Note that the best reply to $AA$ is never $AA$, so the only potential other Nash equilibria are $(AA, AN)$ and $(AN, AA)$. In order for $AA$ to be a best response to $AN$, we require $M > w$. In order for $AN$ to be the best response to $AA$, we need $M/2 > s$. Hence, for $M>w$ and $M/2 > s$, $(AA,AN)$ and $(AN,AA)$ are Nash equilibria.

\paragraph*{8.E.3}
Each firm wants to maximize profits given the type. That is, $q_H, q_L$ must satisfy:
\[ q_L = \arg \max_q \mu q(a - b(q + q_L) - c_L) + (1-\mu) (q(a - b(q + q_H) - c_L))  \]
\[ q_H = \arg \max_q \mu q(a - b(q + q_L) - c_H) + (1-\mu) (q(a - b(q + q_H) - c_H))  \]
The maximization FOCs are:
\[ \mu (a - b(2q_L) - c_L) - \mu q_L b + (1-\mu) (a - b(q_L + q_H) - c_L) - (1-\mu) q_L b = 0 \]
\[ \mu (a - b(q_L + q_H) - c_H) - \mu q_H b + (1-\mu) (a - b(2 q_H) - c_H) - (1-\mu) q_H b = 0 \]
Solving,
\[ a - bq_L - c_L - \mu b q_L - (1-\mu) b q_H = q_L b \]
\[ a - bq_H - c_H - \mu b q_L - (1-\mu) b  q_H  = q_H b \]
\[ \frac{a - c_L}{b} - (2 + \mu)  q_L = (1-\mu)  q_H  \]
\[ \frac{a - c_H}{b} - (3 - \mu) q_H = \mu  q_L   \]
\[ \frac{a - c_H}{b} - \frac{3 - \mu}{1-\mu}\left(\frac{a - c_L}{b} - (2 + \mu)  q_L \right) = \mu  q_L   \]
\[ \frac{a - c_H}{b} - \frac{3 - \mu}{1-\mu}\left(\frac{a - c_L}{b}\right) + \frac{3 - \mu}{1-\mu}(2 + \mu)  q_L  = \mu  q_L   \]
\[\frac{1}{b}\left( \frac{(a - c_H)(1-\mu)}{1-\mu} - \frac{(a - c_L)(3-\mu)}{1-\mu}\right) =\mu q_L - \frac{3 - \mu}{1-\mu}(2 + \mu)  q_L     \]
\[\frac{1}{b}\left( (a - c_H)(1-\mu) - (a - c_L)(3-\mu)\right) =\mu(1-\mu) q_L - (3-\mu)(2 + \mu)  q_L     \]
\[\frac{1}{b}\left( a(1-\mu) - c_H(1-\mu) - a(3-\mu) + c_L(3-\mu)\right) =q_L \left( (\mu-\mu^2)  - (6 + \mu - \mu^2)  \right)    \]
\[\frac{1}{b}\left( -2a - c_H(1-\mu) + c_L(3-\mu)\right) =q_L (-6)   \]
\[q_L = \frac{1}{6b}\left( 2a + c_H-\mu c_H - 3 c_L + \mu c_L\right)     \]
\[q_L = \frac{1}{6b}\left( 2a - 2 c_L + (1-\mu)(c_H - c_L)\right)     \]
\[q_H = \frac{1}{1-\mu}\left(\frac{a - c_L}{b} - (2 + \mu)  q_L \right)   \]
\[ q_H = \frac{1}{1-\mu}\left(\frac{a - c_L}{b} - (2 + \mu) \frac{1}{6b}\left( 2a - 2 c_L + (1-\mu)(c_H - c_L)\right) \right) \]
\[ q_H = \frac{1}{6b(1-\mu)}\left(6a - 6c_L - (2 + \mu)\left( 2a - 2 c_L + (1-\mu)(c_H - c_L)\right) \right) \]
\[ q_H = \frac{1}{6b(1-\mu)}\left(6a - 6c_L - (2 + \mu)2a + (2 + \mu)2 c_L - (2 + \mu)(1-\mu)(c_H - c_L)\right) \]
\[ q_H = \frac{1}{6b(1-\mu)}\left(6a - 6c_L - 4a - 2a \mu + 4c_L + 2 \mu c_L - (2 + \mu)(1-\mu)(c_H - c_L)\right) \]
\[ q_H = \frac{1}{6b(1-\mu)}\left(2a - 2a \mu  - 2c_L + 2 \mu c_L - (2 + \mu)(1-\mu)(c_H - c_L)\right) \]
\[ q_H = \frac{1}{6b(1-\mu)}\left(2a(1- \mu)  - 2c_L(1-\mu) - (2 + \mu)(1-\mu)(c_H - c_L)\right) \]
\[ q_H = \frac{1}{6b}\left(2a  - 2c_L - 2(c_H - c_L) - \mu(c_H - c_L)\right) \]
\[ q_H = \frac{1}{6b}\left(2a  - 2c_H - \mu(c_H - c_L)\right) \]
So we have:
\[ q_L = \frac{1}{6b}\left( 2a - 2 c_L + (1-\mu)(c_H - c_L)\right)     \]
\[ q_H = \frac{1}{6b}\left(2a  - 2c_H - \mu(c_H - c_L)\right) \]
\paragraph*{9.C.2}
Note in this case that Out is not dominated by In$_2$. In pure strategies, the only Nash equilibria is (Out, Fight). This is clearly PBE with the respective belief $\mu_1 \ge 2/3$.

Now, we look for mixed equilibria. First, we note if the incumbent plays a pure strategy of Fight, then since $\gamma < 0$ it is never optimal to play anything other than Out. Similarly, if the incumbent plays a pure strategy Accommodate, then the entering firm's best response is In$_1$, implying the only belief is $\mu_1 = 1$, which does not support the pure strategy of accommodate. So any other equilibria must have the incumbent mix between Fight and Accommodate.

Suppose the incumbent mixes between fighting and accommodating; this is only possible if the expected payoff from accommodating is exactly equal to the expected payoff of fighting $(-1)$. Thus, the belief $\mu_1$ that the incumbent is at In$_1$ must be $2/3$, so this is only possible if player 1 plays a strategy mixing over both In$_1$ and In$_2$. Suppose the incumbent fights with probability $p$. The indifference condition is then:
\[ -p + 3(1-p) = \gamma p + 2(1-p) \ge 0 \]
\[ (\gamma + 1)p = 1 - p \]
\[ (\gamma + 2)p = 1 \]
\[ p = \frac{1}{\gamma + 2} \]
As an incentive check, we confirm that $p \in (0,1)$ and
\[ \gamma p + 2(1-p) = \frac{\gamma}{\gamma + 2} + 2\frac{\gamma + 1}{\gamma + 2} = \frac{3\gamma + 2}{\gamma + 2} \ge 0\]
we note that this requires $\gamma \ge -2/3$.

So, if $\gamma < -2/3$, then the (Out, Fight) equilibrium is the only one we have. If $\gamma > -2/3$, then the expected reward from playing a mixture of In$_1$ and In$_2$ is greater than playing Out, so we get the equilibrium:
\[ \left(\frac{2}{3} \text{In}_1 + \frac{1}{3}\text{In}_2, \frac{1}{\gamma + 2} \text{Fight} + \frac{\gamma + 1}{\gamma + 2} \text{Accommodate} \right) \]
supported by $\mu_1 = 2/3$. Finally, if $\gamma = 2/3$, then the entering firm can also mix on Out, and the equilibria are:\[ \left((1-\alpha) \text{Out} + \frac{2}{3}\alpha \text{In}_1 + \frac{1}{3}\alpha \text{In}_2, \frac{1}{\gamma + 2} \text{Fight} + \frac{\gamma + 1}{\gamma + 2} \text{Accommodate} \right) \]
for $\alpha \in [0,1]$, and belief $\mu_1 = 2/3$.
\paragraph*{9.C.3}
\begin{enumerate}[label=(\alph*)]
\item Consider the second period subgame. By backwards induction, the buyer will accept any offer that is less than $v$. Suppose the seller's prior on the second period subgame is $\mu$ that the player is high type ($v_H$). The seller can offer $v_H - \epsilon$ for expected payoff $\mu (v_H - \epsilon)$, or offer $v_L - \epsilon$ for expected payoff $v_L - \epsilon$. Hence, if $\mu v_H > v_L$, the seller offers $v_H - \epsilon$, and $v_L - \epsilon$ otherwise.

Now, we return to the first period subgame. We first try to find a pooling equilibrium, where both the high type of buyer and low type of buyer will act the same manner. Then if the initial offer is rejected, the seller will have gained no information on the buyer's type and hence the second period belief will be $\mu = \lambda$. Thus, in the second period, the seller will get payoffs $\delta \lambda (v_H - \epsilon)$ or $\delta (v_L - \epsilon)$, whichever is larger. The low type of buyer will receive payoff $\delta\epsilon$ and the high type of buyer will receive payoff $\delta (v_H - v_L + \epsilon)$ if $\lambda v_H \le v_L$, and $\delta \epsilon$ otherwise. Now, in the first period, the low type will want to accept any offer less than $v_L - \delta \epsilon$, since waiting gives no benefit.
If $\lambda v_H \ge v_L$, then the high type will also receive payoff $\delta \epsilon$ after waiting, so the high type will want to accept any offer less than $v_H - \delta \epsilon > v_L - \delta \epsilon$. Hence we cannot have a pooling equilibrium in this case.
If $\lambda v_H < v_L$, then $v_H$ receives payoff $\delta (v_H - v_L + \epsilon)$ in the second period, and would accept any offer less than $v_H - \delta (v_H - v_L + \epsilon)$. But the seller has no incentive to make the separating offer, since
\[ \lambda (v_H - \delta (v_H - v_L + \epsilon)) + (1-\lambda)\delta (v_L - \epsilon) = \lambda (1-\delta) v_H + \delta (v_L + \epsilon) < v_L - \epsilon \]
Hence here, the seller just offers $v_L-\epsilon$ in the first period, and everyone accepts. This is sustained by the off-path belief $\mu = \lambda$ in the second period.

Now, we look for a separating equilibrium. Note that any offer the low type would accept, the high type would also want to accept. Hence in a separating equilibrium, the low type must reject in the first period where the high type accepts. Hence, the seller's belief at the second period must be $\mu = 0$. Given this, the seller offers $v_L - \epsilon$ in the second period, and pockets $\delta (v_L - \epsilon)$ payoff. Now, the high type's defection from the separating strategy (rejecting in period 1) would net a payoff of $\delta ( v_H - v_L + \epsilon)$ if waiting for the second period, so in order for the offer to be separating, the seller must offer at most $v_H - \delta (v_H - v_L + \epsilon) \approx (1-\delta) v_H + \delta v_L > v_L$. Note then the seller's incentive to make this offer requires that this is better than just offering the low type value,
\[ \lambda (v_H - \delta (v_H - v_L + \epsilon)) + (1-\lambda)\delta (v_L - \epsilon) \ge v_L - \epsilon    \]
\[ \lambda (1-\delta)v_H  + \delta (v_L - \epsilon) \ge v_L - \epsilon   \]
\[ \lambda v_H \ge v_L \]
Hence, if $\lambda v_H \ge v_L$, then the seller offers just under $(1-\delta)v_H + \delta v_L$ the first period, and the high type accepts, where the low type rejects, after which the seller offers just under $v_L$ and the low type accepts. This is supported by belief $\mu = 0$ in the second period.
\item Once again, we note that the strategies are monotone in type, that is, the cutoff for accepting an offer is monotone in the type of player. Hence, the belief of the seller in the second round must be uniform on $[\underline{v}, \overline{v}^*]$. Note that in the second round, by backwards induction, the buyer accepts any offer less than the realized type. Hence, the seller maximizes
\[ \max_v v\frac{\overline{v}^* - v}{\overline{v}^* - \underline{v}} \]
So the seller offers $\overline{v}^*/2$ or $\underline{v}$, whichever is larger.

Going back to the first period, suppose the seller offers $v$. Suppose the lowest value that accepts the offer is $v' > v$. Then the offered price next period is the larger of $v'/2$ or $\underline{v}$, which we derived earlier. Then we must have indifference of the buyer between waiting and buying in the first round, which is
\[ v' - v = \delta v'/2 \]
if $v'/2 > \underline{v}$ and
\[ v'- v = \delta (v' - \underline{v}) \]
otherwise. In the first case, we have
\[ v' = \frac{2}{2-\delta}v \]
and in the second we get
\[ v' = \frac{1}{1-\delta}(v - \delta \underline{v}) \]
Note the first case condition then becomes $v > (2-\delta) \underline{v}$, and we have the second case if $v \le (2-\delta)\underline{v}$. Therefore, the seller's  expected payoff from making offer $v > (2-\delta)\underline{v}$ is
\[ \frac{\overline{v} - \frac{2}{2-\delta}v}{\overline{v} - \underline{v}}v + \frac{\frac{2}{2-\delta}v - \underline{v}}{\overline{v} - \underline{v}}\delta\frac{\frac{1}{2-\delta}v}{\frac{2}{2-\delta}v - \underline{v}}\left(\frac{1}{2-\delta}v\right) = \frac{1}{\overline{v} - \underline{v}} \left(\left(\overline{v} - \frac{2}{2-\delta}v \right)v + \frac{\delta v^2}{(2-\delta)^2} \right)  \]
Taking the FOC, we get
\[ \overline{v} - \frac{4}{2-\delta}v + \frac{2\delta}{(2-\delta)^2}v = 0 \]
\[ \overline{v}  = \frac{8 - 6 \delta}{(2-\delta)^2}v \]
\[ v = \frac{(2-\delta)^2 \overline{v} }{8 - 6 \delta} \]
Checking with the case condition, we get
\[ \frac{(2-\delta)^2 \overline{v} }{8 - 6 \delta} > (2-\delta)\underline{v}\]
\[ (2-\delta) \overline{v} > (8 - 6\delta)\underline{v}\]
\[ \overline{v} >\frac{8 - 6\delta}{2-\delta}\underline{v}\]
Now, for the second case, we have $v \le (2-\delta) \underline{v}$. In this case, the expected payoff from making offer $v$ is
\[ \frac{\overline{v} - \frac{1}{1-\delta}(v - \delta \underline{v})}{\overline{v} - \underline{v}}v + \frac{\frac{1}{1-\delta}(v - \delta \underline{v}) - \underline{v}}{\overline{v} - \underline{v}}\delta\underline{v}  \]
\[ = \frac{1}{\overline{v} - \underline{v}}\left(\left(\overline{v} - \frac{1}{1-\delta}(v - \delta \underline{v}) \right)v + \left(\frac{1}{1-\delta}(v - \delta \underline{v}) - \underline{v} \right) \delta \underline{v}  \right) \]
\[ = \frac{1}{\overline{v} - \underline{v}}\left(\left(\overline{v} - \frac{1}{1-\delta}(v - \delta \underline{v}) \right)v + \left(\frac{1}{1-\delta}(v - \delta \underline{v}) - \underline{v} \right) \delta \underline{v}  \right) \]
Taking the FOC, we get
\[ \overline{v} - \frac{1}{1-\delta}(2v - \delta \underline{v}) + \frac{\delta\underline{v}}{1-\delta} = 0 \]
\[ \overline{v} + \frac{2\delta\underline{v}}{1-\delta} = \frac{2}{1-\delta}v \]
\[ v = (1-\delta)\frac{\overline{v}}{2} + \delta\underline{v} \]
Checking consistency with the case condition, we get
\[ (1-\delta)\frac{\overline{v}}{2} + \delta\underline{v} \le (2-\delta) \underline{v}\]
\[ \frac{\overline{v}}{4} \le \underline{v}\]
\[ \overline{v}\le 4\underline{v}\]
Note we have a small overlap region, where
\[ 4 \underline{v} >\overline{v} >\frac{8 - 6\delta}{2-\delta}\underline{v}\]
After some tedious computation, we see that making the higher offer yields better utility. Hence, we have the following equilibrium.If $\overline{v} >\frac{8 - 6\delta}{2-\delta}\underline{v}$, the seller offers
\[ \frac{(2-\delta)^2 \overline{v} }{8 - 6 \delta} \]
in the first period and
\[ \frac{(2-\delta) \overline{v} }{8 - 6 \delta} \]
in the second, and buyers with type greater than
\[ \frac{(2-\delta) \overline{v} }{4 - 3 \delta} \]
buy in the first period and buyers with type less than the above but greater than
\[ \frac{(2-\delta) \overline{v} }{8 - 6 \delta} \]
buy in the second. In the edge case where the optimal solution in the second-stage is non-interior, that is, for $\overline{v} < \frac{8 - 6\delta}{2-\delta}\underline{v}$, then the first stage offer is
\[ (1-\delta)\frac{\overline{v}}{2} + \delta\underline{v} \]
and the second stage offer is $\underline{v}$. Buyers with type greater than $\overline{v}/2$ buy in the first period, and anyone not buying the first period buys in the second.

\end{enumerate}
\paragraph*{9.C.4}
\begin{enumerate}
  \item
  \item
  \item
\end{enumerate}
\section*{Problem 3}
The strategies for the sender are $LL, LR, RL, RR$, denoting the actions after observing $t_1$ and $t_2$. The receiver strategies are $uu, ud, du, dd$, the actions after observing $L$ and $R$. In the subsequent parts, we denote $\mu_L$ the receiver belief in $t_1$ after observing $L$ and $\mu_R$ the receiver belief in $t_1$ after observing $R$.
\paragraph*{(a)}
We start with pooling equilibria. Note that in no case after observing $t_2$ will the sender want to play $L$, so the pooling equilibria must have the sender send $R$ always. In this case, the belief of the receiver on $t_1$ versus $t_2$ is just $(0.5, 0.5)$, and hence $u$ is optimal. Hence, $(RR, uu)$ is a PBE, supported by belief $\mu_L \ge 0.5$ and $\mu_R = 0.5$. Additionally, $(RR, du)$  is also a PBE, supported by $\mu_L \le 0.5$ and $\mu_R = 0.5$. (since the beliefs off-path can be arbitrary).

For separating equilibria, we note that the sender always plays $R$ after realization $t_2$. Then after realizing $t_1$, the separating pure equilibria dictates the sender plays $L$. Hence, the only separating PBE strategy for the sender is $LR$. The only consistent beliefs are then $\mu_R = 0$, $\mu_L = 1$, and the resulting strategy for the receiver is $ud$. So the only separating PBE is $(LR, ud)$ with $\mu_R = 0$, $\mu_L = 1$.

\paragraph*{(b)}
We start with pooling equilibria. If player 1 plays $LL$, then $\mu_L = 1/2$, so player 2 plays $u$ after observing $L$. Off-path, player 2 cannot want to play $d$ after observing $R$, else $t_1$ type would want to deviate to $R$. Hence, a PBE here is $(LL, uu)$, where $\mu_L = 1/2$ and $\mu_R \le 2/3$.

If player 1 plays $RR$, then $\mu_R = 1/2$, so player 2 plays $u$ after observing $R$. But this means the $t_1$ type would rather deviate to $L$, so player 1 cannot play $RR$ in a PBE. Hence the only pooling PBE is the $(LL, uu)$ one we described earlier.

Now, for separating PBE. If player 1 plays $LR$, player 2's beliefs are $\mu_L = 1$, $\mu_R = 0$. So player 2 plays $du$. Note that player 1 does not want to deviate for either type, so this is a PBE: $(LR, du)$ with $\mu_L = 1$, $\mu_R = 0$. Now, if player 1 plays $RL$, player 2's beliefs are $\mu_L= 0$, $\mu_R = 1$, so player 2's best response is $ud$. Once again, we observe that neither type of player 1 would like to deviate. Hence this is also a PBE: $(RL, ud)$ with $\mu_L = 0$, $\mu_R = 1$.

\section*{Problem 4}
First, we find the Nash equilibria. These are $(L,R)$, $(L, pL + (1-p)R)$ and $(M, L)$, where $0 \le p \le 0.7$.

We find a PBE (no restrictions on beliefs off path) first for the first two equilibria. Since player 1 playing $M,R$, is off path, we can allow player 2's belief over the information set to be what we want. If it is $\alpha M + (1-\alpha) R$ for $\alpha < 1/2$, then $(L,R)$ is supported. If it is $(1/2) M + (1/2))R$, then player 2 can take play $pL + (1-p)R$, $p \in [0, 0.7]$ at his information set, since that is optimal given his belief and player 1 playing $L$ is still optimal.

Finally, for $(M,L)$, we note the belief at player 2's information set is $M$ with certainty. Then, $L$ is supported for player 2 under this belief. Hence, this is also a PBE.

\section*{Problem 5}
We first look for Nash equilibria where player 1 plays a pure strategy. Suppose player 1 plays $R$ for sure. Then player 3 will always play $L$ for sure. Then player 2 can mix between $L$ and $R$. Let player 2 play $pL + (1-p)R$. In order for player 1 to not want to deviate, $(1-p) \le 1/4$ or $p \ge 3/4$. Hence, $(R, pL + (1-p)R, L)$ is a Nash equilibrium, for $p \ge 3/4$. Now suppose player 1 plays $L$ for sure. Player 2 will always play $R$, since it always gives a strictly higher payout. Player 3 will then also play $R$ since it gives the highest payout. Hence, $(L,R,R)$ is also a Nash equilibrium.

Finally, we look for Nash equilibria where player 1 mixes. Suppose player 1 plays $(1-a)L + aR$. In order to want to mix, we know that player 1 always gets payoff 1 when playing $R$. So player 1 must have expected payoff at least 1 playing $L$. Let player 2 play $R$ with probability $b$, and player 3 play $R$ with probability $c$. The mixture indifference condition for player 1 implies
\[ 4b(1-c) + bc = 1 \]
\[ 4b - 3bc = 1 \]
\[ b(4 - 3c) = 1 \]
If $b=1$, then we must have $c=1$. If $c=1$, then $b=1$. So if players 2, 3 play $(R,R)$, then player 1 can mix. In order for it to be optimal for 2 to play $R$, the expected value from playing $R$ must exceed the expected value from playing $L$, or
\[ 1 \ge (1-a) + 4a = 1 + 3a \]
This requires $a = 0$, but we already counted $(L,R,R)$. Hence, if player 1 mixes, we must have players 2 and 3 also both mix.

Now, we find the Nash equilibrium where all 3 players mix. We already have the indifference condition for player 1 is :
\[ b(4 - 3c) = 1 \]
The indifference condition for mixing for player 2 is:
\[ 1 = 4ac + a(1-c) = 3ac + a = a(3c + 1) \]
The indifference condition for player 3 is that players 1 and 2 play $(L,R)$ and $(R, L)$ with equal probability, or
\[ a(1-b) = b(1-a) \]
\[ a - ab = b - ab \]
\[ a = b \]
So we get
\[ a(4 - 3c) = 1 \]
\[ a(3c + 1) = 1 \]
This implies
\[ 4 - 3c = 3c + 1 \]
\[ c = 1/2 \]
\[ a = 2/5 \]
\[ b = 2/5 \]
Hence, the other Nash equilibrium is
\[ \left( \frac{3}{5}L + \frac{2}{5}R,\frac{3}{5}L + \frac{2}{5}R,\frac{1}{2}L + \frac{1}{2}R \right) \]

Now, for PBE, we need to construct beliefs for players 2 and 3 that support the equilibrium. Consider $(R, pL + (1-p)R, L)$, for $p \ge 3/4$. To support this, we have player 2's belief be that player 1 played $R$ for sure, and player 3's belief is that players 1 and 2 played $(R, L)$. Alternatively, for $p = 0$, the equilibrium $(R, R, L)$ is also supported for player 2 believing player 1 played $R$ and player 3 believing players 1 and 2 played some mixture of $L,L$ and $R,L$ (this is fine since player 3's belief is off-path). Similarly, for $(L,R,R)$, we have player 2's belief is that player 1 played $L$ and player 3's belief is that players 1 and 2 played $(L,R)$.

Now, for the mixed-strategy Nash equilibrium, we can make this a PBE by the following beliefs: player 2 must believe with probability $3/5$ that player 1 played $L$, and player 3 has the following belief:
\[ \frac{3}{7}(L,L) + \frac{2}{7}(L,R) + \frac{2}{7}(R,L) \]
\section*{Problem 6}
For pure-strategy Nash equilibria, we note that we have $(R, C, L)$ and $(R, S, R)$. Note that player 1 guarantees maximum payoff by playing 1: the only way player 1 could mix is if players 2 and 3 always play $C, L$. Clearly, if player 3 plays $L$, player 2 is best off playing $C$. Further, in order for player 3 to want to play $L$, we need player 1 playing $R$ with higher probability than $L$. Hence, $(\alpha L + (1-\alpha) R, C, L)$ is Nash for $\alpha \le 1/2$. If player 1 doesn't mix, the only way for player 2 to want to mix is if player 3 also mixes, but player 3 only wants to mix when player 2 plays $S$. Hence, for players 1 and 2 playing $R, S$, player 3 can play $\beta L + (1-\beta)R$ with $\beta \le 1/2$, and this will be a Nash eq.
Hence $(R, S, \beta L + (1-\beta)R)$ with $\beta \le 1/2$ are the other Nash equilibria.

Now, for PBE, we need beliefs of players 2 and 3 to support the strategies. Let $\mu_2$ be the belief of player 2 in the left side, and $\mu_3$ be the belief of player 3 in the left side. For $(R,C,L)$, this is supported by $\mu_2 = 0, \mu_3 = 0$. We can also support $(R,S,R)$ with $\mu_2 = 0$, $\mu_3 \ge 1/2$ (in order for player 3 to play $R$).
Further, $(\alpha L + (1-\alpha) R, C, L)$ for $\alpha \le 1/2$ is supported for $\mu_2 = \alpha$, $\mu_3 = \alpha$. Lastly, $(R, S, \beta L + (1-\beta)R)$ with $\beta \le 1/2$ is supported by beliefs $\mu_2 = 0, \mu_3 = 1/2$.
\end{document}
	% line of code telling latex that your document is ending. If you leave this out, you'll get an error
