%Jennifer Pan, August 2011

\documentclass[10pt,letter]{article}
	% basic article document class
	% use percent signs to make comments to yourself -- they will not show up.

\usepackage{amsmath}
\usepackage{amssymb}
\usepackage{enumitem}
	% packages that allow mathematical formatting

\usepackage{graphicx}
\usepackage{tikz}
	% package that allows you to include graphics

\usepackage{setspace}
	% package that allows you to change spacing

\onehalfspacing
	% text become 1.5 spaced

\usepackage{fullpage}
	% package that specifies normal margins


\begin{document}
	% line of code telling latex that your document is beginning


\title{ECON501 Problem Set 2}

\author{Nicholas Wu}

\date{Spring 2021}
	% Note: when you omit this command, the current dateis automatically included

\maketitle
	% tells latex to follow your header (e.g., title, author) commands.

\section*{Problem 2}
First, we note that $B$ is strictly dominated by $A$ for the row player ($(10, -4, 5, 0) \gg (9, -6, 3, -2)$). Eliminating $B$, we have that $a$ is dominated by $b$ for the column player, since $(5, 6, -2, 3) \gg (2, 0, -3, 1)$. Among the remaining strategies, $A$ and $C$ are both dominated by $E$ since $(-4, 5, 0) \ll (-2, 6, 2)$ and $(-4, 5, 0.1) \ll (-2, 6, 2)$.

Now, the remaining strategies for the row player are $D$ and $E$, and the remaining column strategies are $b, c, d$. However, we note that the strategy $0.4b + 0.6d$ dominates $c$ since $(-0.8, 1.2) \gg (-1, 1)$. Hence $c$ is strictly dominated.

The remaining strategies $D, E$ and $b, d$ cannot be eliminated further, so these remain.

\section*{Problem 3}
We first argue playing $0$ is weakly dominated by playing 1. Playing 0 guarantees payoff 0 in any situation. Playing 1 guarantees payoff 1 in any situation (if the other person plays less than 100, then you get 1, and if the other person plays 100, you still get 1). Hence 0 is a weakly dominated strategy, so we eliminate it.

We now argue that the strategy of playing $100$ is weakly dominated by the strategy of playing $99$. If the other player also plays $100$, then $99$ achieves a better payoff ($99 > 50$). Suppose the other player played $99$. Then again playing 99 gives a better payoff than 100, because $50 > 1$. If the other player played $k \in [2, 98]$, then the player gets $100-k$, so playing 99 does equally as well as 100 here. Finally, if the other player played $1$, playing 100 would get payoff 99, and playing $99$ also ensures payoff 99, so for $k \in [1, 98]$ the player does the same playing 99 as 100, and for $k \in [99, 100]$ the player does better playing 99 than playing 100. Hence $100$ is weakly dominated.

We now can iterate this argument. Suppose we have not yet eliminated the strategies from $[50 - x, 50 + x]$, where $x$ is a positive integer. Consider playing $50 - x + 1$ instead of $50 - x$. This guarantees a payoff of at least $50 - x + 1$, since if the other person plays any number less than $50 + x - 1$ the sum is $\le 100$, and the player gets $50 - x + 1$, and if the other person plays $50 + x$, $50 -x + 1$ is the lower bid and hence the player always gets $50 - x + 1$. This is strictly better than playing $50 - x$, which guarantees payoff $50 - x$, so $50 - x$ is weakly dominated by $50 - x + 1$.

Now, after eliminating $50 - x$, we can show $50 + x$ is weakly dominated by $50 + x - 1$. Suppose the other player plays $50 + x$. Playing $50 + x - 1$ guarantees payoff $50 + x - 1 \ge 50$, so playing $50  + x - 1$ is weakly better  in this case. If  the other player plays $50 + x - 1$, then playing $50 + x - 1$ guarantees payoff $50 \ge 50 - x + 1$, so $50 + x - 1$ is also weakly better. If the other player plays anything else from $50 - x + 2$ to $50 + x - 2$, they will have the lower bid and the sum in either case will be greater than 100, so there is no change in payoff from switching to $50 + x - 1$ from $50 + x$.
Lastly, if the other player plays $50 - x + 1$, switching to $50 + x - 1$ makes the sum $100$, but in either case the payoff is $50 + x - 1$, so there is no change in payoff still here. Thus, $50 - x$ is weakly  dominated by $50 - x + 1$, and now the range of remaining strategies is $[50 - x + 1, 50 + x - 1]$

We can apply our argument iteratively; the only strategy remaining will be playing $50$.

\section*{Problem 4}
Suppose player $1$ plays $pU + (1-p)D$, and player 2 plays $qL + (1-q)R$. The payoffs of player 3's actions are then:
\begin{center}
\begin{tabular}{  c | c  }
 A & $ 9pq $ \\ \hline
 B & $ 9p(1-q) + 9(1-p)q = 9p + 9q - 18pq$ \\ \hline
 C & $ 9(1-p)(1-q) = 9 - 9p - 9q + 9pq$ \\ \hline
 D & $ 6pq + 6(1-p)(1-q) = 6 - 6p - 6q + 12pq$
\end{tabular}
\end{center}
Suppose, for sake of contradiction, D is the best reply. For D to be better than A, we have
\[ 6 - 6p - 6q + 12pq > 9pq \]
\[ 2 - 2p - 2q + pq > 0 \]
\[ \frac{1}{2}pq + 1 > p + q \]
\[ p - \frac{1}{2}pq  < 1 - q \]
\[ p\left(1 - \frac{1}{2}q \right) < 1 - q \]
\[ p < \frac{1-q}{1 - 0.5q} \]
For D to be better than C,
\[ 6 - 6p - 6q + 12pq > 9 - 9p - 9q + 9pq \]
\[ p + q > 1 - pq \]
\[ p + pq > 1 - q \]
\[ p(1+q) > 1-q \]
\[ p > \frac{1-q}{1+q} \]
For D to be better than B,
\[ 6 - 6p - 6q + 12pq > 9p + 9q - 18pq \]
\[ 6 - 15p - 15q + 30pq > 0 \]
\[ (2/5) - p - q + 2pq > 0 \]
\[ \frac{2}{5} + 2pq > p + q \]
\[ p - 2pq < \frac{2}{5} - q \]
\[ p(1-2q) < \frac{2}{5} - q \]
For $q < 1/2$, this implies
\[ p < \frac{0.4 - q}{1 - 2q}\]
But from a prior condition,
\[ p > \frac{1-q}{1+q} \]
But on the region $q \in (0, 1/2)$,
\[ \frac{1-q}{1+q} > \frac{0.4 - q}{1 - 2q} \]
So we have a contradiction.

Similarly, if $q > 1/2$,
\[ p(1-2q) < \frac{2}{5} - q  \]
\[ p > \frac{0.4 - q}{1 - 2q} \]
But
\[ p < \frac{1-q}{1 - 0.5q} \]
And on $q \in (1/2, 1)$
\[ \frac{0.4 - q}{1 - 2q}  >  \frac{1-q}{1 - 0.5q} \]
Hence we have a contradiction.

Lastly, if $q = 1/2$, then the payoffs become
\begin{center}
\begin{tabular}{  c | c  }
 A & $ 4.5p $ \\ \hline
 B & $ 4.5 $ \\ \hline
 C & $ 9(1-p)(1-q) = 4.5 - 4.5p$ \\ \hline
 D & $ 6pq + 6(1-p)(1-q) = 3$
\end{tabular}
And B is strictly better than D, a contradiction again.
\end{center}

Therefore, D cannot be a best reply, no matter what values $q$ takes. However, $D$ is not dominated. Consider any arbitrary other strategy $pA + qB + rC + (1-p-q-r)D$. In order for this to dominate $D$, it must be better than $D$ in every situation. The payoffs of this strategy for $(U, L)$ are:
\[ 9p + 6(1-p-q-r) = 6 + 3p - 6q - 6r \]
For $(U, R)$ and $(D,L)$, the payoffs are just
\[ 9q \]
For $(D, R)$ the payoff is
\[ 9r + 6(1-p-q-r) = 6 - 6p - 6q + 3r \]
We need these to be better than $D$ in each case, so
\[ 6 + 3p - 6q - 6r \ge 6 \]
\[ 9q \ge 0 \]
\[ 6 - 6p - 6q + 3r \ge 6 \]
From the first and third inequalities, we get
\[ p \ge 2q + 2r \]
\[ r \ge 2p + 2q \]
\[ p + r \ge 2(p + r) + 4q \]
But then we get
\[ p + r + 4q \le 0 \]
which is only possible if $p = r = q = 0$, which is just $D$ itself. Hence no other strategy dominates $D$.

\section*{Problem 5}
\begin{enumerate}[label=(\alph*)]
\item If either player has a strategy guaranteeing a win, then that player $i$ by definition has a maximin strategy $s_i$ such that $u_i(s_i, s_j) \ge 1$ for all $s_j$. Since the utility can only be $0$ or $\pm 1$, this implies $u_i(s_i, s_j) = 1$, for any $s_j$. Hence $s_i$ weakly dominates any other strategy, and so $D(N)$ has $u_i$ constant and equal to 1, and hence $u_j$ is constant and equal to $-1$, so this is dominance solvable in one step.
\item Now, assume that neither player can guarantee a win, and hence each player can enforce 0. \begin{enumerate}[label=(\roman*)]
  \item Fix the strategy profile $(x_1, x_2)$ and define $x_{h^*}$ as in the problem. By definition, $v_1(x_{h^*}) \neq 0$, so it is either 1 or $-1$. Suppose, for sake of contradiction, that it is $1$. Then for the subgame $\Gamma(x_{h^*})$, there exists some strategy $s_1'$ within this subgame that is winning, i.e. achieves $1$, since $v_1(x_{h^*}) = 1$ by our assumption. Consider the alternative strategy
  \[ s^*_1(x) = \begin{cases}
    s_1(x) & x \not \in X(x_{h^*}) \\
    s'_1(x) & x \in X(x_{h^*})
  \end{cases} \]
  Then clearly $s^*_1$ does equally as good as $s_1$ if the subtree $x_{h^*}$ is never reached, and since it achieves $1$ in that subtree, it does weakly better in that subtree. Hence $s^*_1$ dominates $s_1$ in $N$, a contradiction. Therefore $v_1(x_{h^*}) = -1$.
  \item If it were player 2's turn at $x_{h^* - 1}$, then player 2 could take an action that would force $1$ to $x_{h^*}$, and we know  that since $v_1(x_h^*) = -1$, this is optimal for player 2. That implies that $v_1(x_{h^*-1}) = -1$, which is false, since we know by construction that $v_1(x_{h^* - 1}) = 0$. Hence it must be player 1's turn at $x_{h^* - 1}$.
  \item \begin{enumerate}[label=(\Alph*)]
    \item Suppose not, that some $s_2' \in D(N)$ is such that $u_1(s_1, s_2') > u_1(s_1', s_2')$. Note the paths dictated by $s_1, s_2'$ and $s_1', s_2'$ must reach $x_{h^*-1}$, since $s_1'$ is equal to $s_1$ outside $\Gamma (x_{h^*-1})$ and $u_1(s_1', s_2') \neq u_1(s_1, s_2')$.
    Since $s_1'$ is maximin on $\Gamma (x_{h^*-1})$, and since $v_1(x_{h^*-1}) = 0$, we have $u_1(s_1', s_2') \ge 0$, which means that $u_1(s_1, s_2') = 1$. Recall that $v_1(x_{h^*}) = -1$, as proved before, so $v_2(x_{h^*}) = 1$. Pick a winning strategy $s_2^*$ in $\Gamma(x_{h^*})$.
    Consider $s_2^*$ such that
    \[ s_2''(x) = \begin{cases}
      s_2^*(x) & x \not \in X(x_{h^*}) \\
      s_2'(x) & x \in X(x_{h^*})
    \end{cases} \]
    Clearly, since $s''_2$ equals $s_2'$ outside of $\Gamma(x_{h^*})$, it does equally well outside $\Gamma(x_{h^*})$. However, since $s_2^*$ is winning in $\Gamma(x_{h^*})$, and $u_2(s_1, s_2') = -1$, $u_2(s_1, s_2'') > u_2(s_1, s_2')$, and hence $s_2'$ is dominated by $s_2''$ in $N$, so we have a contradiction that $s_2' \in D(N)$.
    Hence $s_1'$ is at least as good as $s_1$ against any strategy in $D(N)$.
    \item When player 2 plays $s_2$, we note that $s_1'$ and $s_1$ will both reach $\Gamma(x_{h^*-1})$. This implies by construction of $s_1'$, $u_1(s_1', s_2) = 0$, since $s_1'$ plays maximin in $\Gamma(x_{h^*-1})$ and $v_1(x_{h^*-1}) = 0$. However, $u_1(s_1, s_2) = -1$, so $s_1$ is dominated by $s_1'$, and hence is dominated in $D(N)$.

  \end{enumerate}
\end{enumerate}
\end{enumerate}
\end{document}
	% line of code telling latex that your document is ending. If you leave this out, you'll get an error
